\documentclass[linguex]{lfg-proc}

%===========================================================
% Fill in your title, author names, and affiliations here

\title{Your title here}
\author{Author 1 \affiliation{Author 1's affiliation}%
  % \and Author 2 \affiliation{Author 2's affiliation}
  % \and Author 3 \affiliation{Author 3's affiliation}
  % \and Author 4 \affiliation{Author 4's affiliation}
}

%===========================================================
% Put your own package calls and macros here

% \usepackage{expex}
\usepackage{pifont}
% \newcommand{\tuple}[1]{\ensuremath{\langle #1 \rangle}}
\newcommand{\goodex}{\makebox[0pt][r]{\normalfont\ding{51}\ignorespaces}}
\newcommand{\badex}{\makebox[0pt][r]{\normalfont\ding{55}\ignorespaces}}

%===========================================================

\begin{document}
\maketitle

%===========================================================
% Put the abstract here

\begin{abstract}
Your short ($\leq$150 words) abstract goes here.
\end{abstract}

%===========================================================
% Body text here

\section{Introduction}

This document serves as a style guide for papers submitted to the LFG
Proceedings, and also as an example\slash template to be used by authors.

\section{Style}

\subsection{Title page}
\begin{itemize}
  \item Please use sentence capitalisation in your heading; i.e., don't
        capitalise every content word. So \Next is correct, while \NNext is not:

        \ex. \goodex\textbf{A very interesting paper}

        \ex. \badex\textbf{A Very Interesting Paper}

\end{itemize}

\subsection{Acknowledgements}
\begin{itemize}
  \item Use the \verb=\acknowledgements= command to add an acknowledgements
        footnote after your first sentence.\acknowledgements{I thank X and Y.}
\end{itemize}

\subsection{Page numbering}
\begin{itemize}
  \item Do not put page numbers on your paper -- this will be done
        automatically.
\end{itemize}


\subsection{Referencing}

\begin{itemize}
  \item We use Bib\TeX\ with the \verb=natbib= package (Bib\LaTeX/biber support
        might come in the future \dots).
  \item Use \verb=\citet= when mentioning the reference in a running text:
        ``\citet{Dalrymple2015} deals with morphology in LFG''.
        \begin{itemize}
          \item Use \verb=\citealt= to omit the parentheses when the reference
                is already inside parentheses: ``LFG covers all aspects of
                linguistic stucture (and this includes morphology, as
                \citealt{Dalrymple2015} shows)''.
          \item Note that we \textbf{don't} follow the practice of referring to
                \emph{works} without parentheses and \emph{authors} with --
                there are too many fuzzy cases.
        \end{itemize}
  \item Use \verb=\citep= for parenthetical references: ``Morphology is easy to
        deal with in LFG \citep{Dalrymple2015}''.
  \item Both commands take an optional argument for page numbers:
        ``\citet[66]{Dalrymple2015} illustrates the split between p-form and
        s-form \citep[177]{dalrymple-mycock:prosody}''.
\end{itemize}

\subsection{Bibliography}

\begin{itemize}
  \item If you use the \verb=lfg-sp.bst= style file, everything should Just
        Work.
        \begin{itemize}
          \item In general, this follows the LSA's
                \href{https://www.linguisticsociety.org/resource/unified-style-sheet}{Unified
                Style Sheet}.
            \end{itemize}

  \item See the included \verb=sample.bib= for examples of how to format bib
        entries.

  \item We also include the \verb=lfg-master.bib=, which was put together for
        the \textit{Handbook of Lexical Functional Grammar}
        \citep{lfg-handbook}, as a reference and a source to make it easy to
        copy-paste \verb=.bib= entries from.

  \item \emph{However}, many of the entries in \verb=lfg-master.bib=, in
        particular those for LFG Proceedings papers, do not include links. We
        ask that you \emph{do} include links in your \verb=.bib= entries
        whenever you can, and that you prefer DOIs over more brittle URLs where
        possible.
        \begin{itemize}
          \item If your entry includes both a DOI and a URL, only the DOI will
                be displayed.
          \item Note that the contents of the Bib\TeX\ \verb=doi= field should not
                include the full URL; that is, write \Next not \NNext in your
                \verb=.bib= file entry:

                \ex. \goodex\ \texttt{doi = \{10.5281/zenodo.10037797\}}

                \ex. \badex\ \texttt{doi = \{https://doi.org/10.5281/zenodo.10037797\}}

        \end{itemize}

  \item \textbf{Note:} if (and only if) you wish to use the Chinese and Japanese
        fonts included in some entries in the \texttt{lfg-master.bib} file, you
        will need to do the following:
        \begin{itemize}
          \item Add \verb=\includepackage{fontspec}= to your preamble.
          \item Compile with \texttt{lualatex} or \texttt{xelatex} (not \texttt{pdflatex}).
          \item Include the following definitions in your preamble:\\
          \verb=\newfontfamily\cn[]{SourceHanSerifTC} % Chinese font=
          \verb=\newfontfamily\jpn[]{SourceHanSerifJP} % Japanese font=
          \item If you do not already have the Source Han Serif fonts, you will also need to install them from here: \\
          \url{https://source.typekit.com/source-han-serif/#get-the-fonts}
        \end{itemize}
\end{itemize}

\subsection{Linguistic examples}

\begin{itemize}
  \item By default, we provide the \href{https://ctan.org/pkg/linguex}{linguex}
  package to format examples. This has a simple syntax: use \verb=\ex.= to
  introduce a new example, \verb=\a.= to introduce a new sub-list, and
  \verb=\b.= for subsequent sub-examples (you can also use \verb=\c.=,
  \verb=\d.=, etc., but these do exactly the same thing as \verb=\b.=):

        \ex. Here is an example.

        \ex. Here is an example that has sub-examples.%
        \a. Like this.%
        \b. And this.%
        \c. And so on.%
        \b. *And so fourth.\footnote{Note the proper alignment of the judgement
        marker *. This is handled automatically by \texttt{linguex} for *, \#,
        and \%.}

  \item To include a three-line gloss, simply append the relevant example
        command with a \verb=g= (\verb=\exg.=, \verb=\ag.=, etc.); then
        start each new line with a \verb=\\= (see the source code of the
        following example for clarification):

        \exg.%
        Voici un example avec une glose.\\
        here.is a.\textsc{masc} example with a.\textsc{fem} gloss\\
        `Here is an example with a gloss.' \exlang{French}

  \item In addition to the usual \verb=\label{...}= and \verb=\ref{...}= ways of
        referring to examples, \verb=linguex= also provides the commands
        \verb=\Next=, \verb=\NNext=, \verb=\Last=, and \verb=\LLast= to easily
        refer to the next, next but one, last, and one before last examples (but
        note that these references will not be clickable in the same way as
        those created using a standard \verb=\label{...}=+\verb=\ref{...}= pair
        will be).

  \item One peculiarity of \verb=linguex= is that you must leave a blank line
        after the end of an \verb=\ex.=, otherwise \LaTeX\ will throw an error.

  \item Please see
        \href{https://ctan.uib.no/macros/latex/contrib/linguex/doc/linguex-doc.pdf}{the
        \texttt{linguex} documentation} for full details of the package.

  \item Since \verb=\hfill= does not behave normally on the first (or second)
        line of a glossed example, it can be difficult to display the sort of
        right-aligned notes which are often used to identify the language of an
        example. To solve this problem, the \verb=lfg-proc= class (when called
        with the \verb=[linguex]= option, as it is by default) defines a command
        \verb=\exlang{...}=: when used at the end of the third line of a glossed
        example, this displays its argument as a right-aligned parenthetical
        note on the first line of the example (as shown above for French).

  \item You are free to use your own example package of choice instead of
        \verb=linguex= if you prefer: simply remove the \verb=[linguex]= option
        from the call to the \verb=lfg-proc= class at the start of this file and
        then add an appropriate \verb=\usepackage{...}= as usual. Please make
        sure that your examples look as close as possible to those shown here,
        however.

  \item \verb=lfg.cls= also loads the \verb=tipa= package by default, which can
        be used to typeset IPA characters straightforwardly:
        \textipa{/lAikDIs/}. If for whatever reason you really don't want to
        load \verb=tipa=, simply pass the option \verb=notipa= to the
        \verb=\documentclass= command at the top of this file.

\end{itemize}


\subsection{Structure}

\subsubsection{Subsections}

Don't use anything smaller than a \verb=\subsubsection=.


%===========================================================
% References

\bibliographystyle{lfg-sp}
\bibliography{sample}

%===========================================================
\end{document}
